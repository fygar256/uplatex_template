\documentclass[uplatex]{jsarticle}
\usepackage{amsmath,amssymb}
\usepackage{txfonts}
\usepackage{bbm}
\parindent = 0pt
\begin{document}
\gt
\fontsize{15pt}{22pt}\selectfont

回転行列を求める。
\\
基本ベクトル
$\vec{e}=(1,0),\vec{f}=(0,1)$
を、角$\theta$回転すると、\\
$\vec{e'}=\left(\cos\theta,\sin\theta\right),\vec{f'}=\left(-\sin\theta,\cos\theta\right)$
になる。
\\


$\vec{p}=x\vec{e}+y\vec{f}$であった点は、
$\vec{p'}=x\vec{e'}+y\vec{f'}$に移る。\\
これを計算すると、\\
$\vec{p'}=x(\cos\theta,\sin\theta)+y(-\sin\theta,\cos\theta)\\
=(x\cos\theta-y\sin\theta,x\sin\theta+y\cos\theta)$.
\\

$\vec{p'}=(X,Y)$とすると、連立式\\

$
\left\{
\begin{array}{l}
X=x\cos\theta-y\sin\theta\\
Y=x\sin\theta+y\cos\theta
\end{array}
\right.
$
\\
を、行列の形に直して、
\\
\\
$
  \left(\begin{array}{rr}X \\ Y \end{array}\right)=
  \left(
    \begin{array}{rr}
      \cos\theta & -\sin\theta \\
      \sin\theta &  \cos\theta
    \end{array}
    \right)
  \left(\begin{array}{rr}x \\ y \end{array}\right)
$
\\
\\
よって、角$\theta$の回転行列$R$は、\\
$
R(\theta)=
  \left(
    \begin{array}{rr}
      \cos\theta & -\sin\theta \\
      \sin\theta &  \cos\theta
    \end{array}
    \right)
$
となる。

\end{document}
